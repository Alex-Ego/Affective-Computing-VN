\chapter{Background}
\label{ch2}
Technology in the past decades has been advancing exponentially. So much, in fact, that we can relegate data analysis to them for better accuracy and reliability than what a human can possibly achieve. This is what is called as Machine Learning (sometimes referred only as ML).
There is a variety of scenarios where it is useful, such as pattern recognition, which relates extensively to most of this project's work.\\
In this chapter, some key concepts will be explained for easier comprehension of this thesis and the project itself as a whole.

\section{Supervised Machine Learning}
Supervised ML can be described, broadly and figuratively speaking, as a black box where some data is inserted as an input and numbers come out of it as an output \citep{rf8}. This output, as opposed to other types of Machine Learning, is later analyzed and compared to real life data. This means this type of ML is best used as an auxiliary tool more than the main source of information from a process.
Some more advanced models of ML allow some internal parameters inside this figurative black box to be able to be tampered with, so that some characteristics of the input data can have effect on the output, these parameters are called \textit{weights} \citep{rf9}.
Most ML algorithms have two stages: training and validation:
\begin{itemize}
\item Training processes the inputs and makes educated guesses, and in case of guessing incorrectly, depending on the obtained result, the weights are changed accordingly.
\item Validation is as simple as it sounds, some input is fed to the algorithm and information needs to be compared to the real results to test the accuracy percentage.
\end{itemize}
One of these models that is one of the most used nowadays is the one called \textit{Neural Network}.

\subsection{Neural Network}
A neural network works by using \textit{neurons} that utilize layers that individually weigh the input given to them from the initial text or, if this has been processed already, from another neuron \citep{rf9}.
Likewise, similar to how biological brains work, these algorithms can only predict reliably if given enough data to train and validate their outputs with.

\section{Sentiment Analysis}
Sentiment Analysis (or Opinion Mining, as it is also known) as a tool for data analysis is arguably a recent happening. The term was coined in 2003 and has evolved ever since \citep{rf3}.
This type of data analysis has a lot of potential usages that have yet to be implemented in the daily life.

\subsection{Concept}
The specific execution of the algorithm varies depending on the intended purpose, but the concept and process that is used is generally the same:
\begin{itemize}
	\item The sentence to analyze is broken down to its component parts, this process is called \textit{tokenization}, and the resulting products are called, fittingly, \textit{tokens}.
	\item Every token is then tagged, making it part of an internal dictionary or \textit{lexicon}
	\item A score is assigned to every token depending on the used dataset.
\end{itemize}
The end score could be left as-is or can be reintroduced to the algorithm for a multi-layered approach depending on its focus \citep{rf4}.

\subsection{Tokenizing}
Tokenizing is the process that happens while making tokens, the way it works is very straightforward: every word in the lexicon that a machine can read is assigned a number for easier reading. Taking the following example:
\begin{center}
\fbox{This is an example text}
\end{center}

We can tell there are 5 words in the example phrase. So the tokenizing process would make the example look in the following way:
\begin{center}
\fbox{1,	2,	3,	4,	5}
\end{center}

\noindent where 1 corresponds to the word ``This'', 2 corresponds to ``is'', 3 to ``an'' and so on.

The interesting part about this process would happen if we used another example phrase, like the following:
\begin{center}
\fbox{This is another example}
\end{center}

If we did the tokenization process, it would be processed in this way:
\begin{center}
\fbox{1,	2,	6,	4}
\end{center}

Since the internal lexicon already knows some of the words in this second example, it reuses their token, adding new ones (in this example, ``another'' is 6) if needed.\\

This is fairly useful for a machine learning algorithm, since it will not have to compare such massive amount of characters in a string each time, and it would only need to evaluate integers. Whether its focus is either frequency or comparison.

\clearpage