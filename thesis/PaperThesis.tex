%%%%%%%%%%%%%%%%%%%%%
% Documento maestro %
%%%%%%%%%%%%%%%%%%%%%
\documentclass{fime}

%%%%%%%%%%%%%%%%%%%%%%%%%%%%%%%%%%%%%%%%%%%
% Cargando paquetes y definiendo opciones %
%%%%%%%%%%%%%%%%%%%%%%%%%%%%%%%%%%%%%%%%%%%
% Aquí puedes cargar los paquetes que vas a usar. La clase
% fime ya incluye babel, inputenc, graphicx y los de la AMS.
% Cargar un paquete está a tu libertad (y responsabilidad).
\usepackage{hyperref}
\usepackage{graphicx}
\usepackage{pdflscape}
\usepackage{amsmath}
\usepackage[utf8]{inputenc}
\usepackage[numbers,sort&compress]{natbib}
\usepackage[bottom]{footmisc}
\usepackage{float}
\usepackage{footmisc}
\usepackage{adjustbox}
\usepackage{tikz}
\usepackage{url}
\usepackage{listings}

\setlength{\footnotesep}{0.5cm}
\setlength\footnotemargin{3pt}

    \hypersetup{breaklinks=true,colorlinks=true,
        linkcolor=black,citecolor=black,urlcolor=black}

%%%%%%%%%%%%%%%%%%%%%
% Definiendo campos %
%%%%%%%%%%%%%%%%%%%%%
\def\titulo{Sentiment Analysis through a chatbot}
\def\autor{Alexander Espronceda Gómez}
\def\matricula{1742000}
\def\grado{Ingeniería en Tecnología de Software}
% En caso de que el grado tenga orientación o especialidad llenar el siguiente
% campo dejando un ESPACIO INICIAL, en caso contrario, dejar vacío
\def\orientacion{}
\def\fecha{Enero 2022} % Coloca el mes con mayúscula inicial

\def\asesor{Dra. Satu Elisa Schaeffer}
\def\revisorA{Dra. Sara Elena Garza Villarreal}
\def\revisorB{Dr. Romeo Sánchez Nigenda}
% En el caso de que tu tesis sea de doctorado activa la variable cambiándola a \doctoradotrue
% y define tus otros dos revisores
\newif\ifdoctorado\doctoradofalse
\def\revisorC{Nombre del revisor C}
\def\revisorD{Nombre del revisor D}
% El visto bueno siempre va
\def\vobo{Dr. Fernando Banda Muñoz}

%%%%%%%%%%%%%%%%%%%%%%%
% Inicia el documento %
%%%%%%%%%%%%%%%%%%%%%%%
\begin{document}

\frontmatter
\pagestyle{main}

%%% Incluye PortillasM si tu tesis es de Maestría
%%% y PortillasD si es de doctorado.
% Portadas (Maestría)

\def\uanl{Universidad Autónoma de Nuevo León}
\def\fime{Facultad de Ingeniería Mecánica y Eléctrica}
\def\depg{Subdirección de Estudios de Posgrado}
\def\snnl{San Nicolás de los Garza, Nuevo León}

%%%%%%%%%%%%%%%%%%%%%%%%
% Primer portada: UANL %
%%%%%%%%%%%%%%%%%%%%%%%%
\thispagestyle{empty}

\begin{scshape}
\begin{center}
	{\Large \uanl} \\[5mm]
	{\large \fime} \\[5mm]
	{\large \depg}
	\vskip 15mm
	\includegraphics[height=55mm]{uanl}
	\vskip 12mm
	\begin{tabular}{p{11cm}}
		\centering
		{\large \titulo}
	\end{tabular}
	\vskip 7mm
	{por}\\[7mm]
	{\large \autor}\\[7mm]
	{como requisito parcial para obtener el grado de}\\[3mm]
	\MakeUppercase{\grado}\\
	\orientacion
	\vfill
	\fecha
\end{center}
\end{scshape}

%%%%%%%%%%%%%%%%%%%%%%%%%
% Segunda portada: FIME %
%%%%%%%%%%%%%%%%%%%%%%%%%
\newpage
\thispagestyle{empty}

\begin{scshape}
\begin{center}
	{\Large \uanl} \\[5mm]
	{\large \fime} \\[5mm]
	{\large \depg}
	\vskip 16mm
	\includegraphics[height=50mm]{fime}
	\vskip 16mm
	\begin{tabular}{p{11cm}}
		\centering
		{\large \titulo}
	\end{tabular}
	\vskip 7mm
	{por}\\[7mm]
	{\large \autor}\\[7mm]
	{como requisito parcial para obtener el grado de}\\[3mm]
	\MakeUppercase{\grado}\\
	\orientacion
	\vfill
	\fecha
\end{center}
\end{scshape}

%%%%%%%%%%%%%%%%%%%%%%%%%%%%%
% Carta del comité de tesis %
%%%%%%%%%%%%%%%%%%%%%%%%%%%%%
\newpage
\thispagestyle{empty}
\enlargethispage{5mm}

\begin{center}
{\bf \large \uanl} \\
{\bf \fime} \\
{\bf \depg}
\end{center}
\vskip 4mm

Los miembros del Comité de Tesis recomendamos que la Tesis <<\titulo>>, realizada por el alumno \autor, con número de matrícula \matricula, sea aceptada para su defensa como requisito parcial para obtener el grado de \grado\orientacion.
\ifdoctorado{\vskip 10mm}\else{\vskip 8mm}\fi

\begin{center}
El Comité de Tesis\\
\ifdoctorado{\vskip 15mm}\else{\vskip 25mm}\fi

\ifdoctorado{%%%
\begin{tabular}{p{37mm}p{21mm}p{12mm}p{21mm}p{37mm}}
	\cline{2-4}
	& \multicolumn{3}{c}{\asesor} & \\
	& \multicolumn{3}{c}{Asesor}  & \\[15mm]
	\cline{1-2} \cline{4-5}
	\multicolumn{2}{c}{\revisorA} & & \multicolumn{2}{c}{\revisorB} \\
	\multicolumn{2}{c}{Revisor}   & & \multicolumn{2}{c}{Revisor}   \\[17mm]
	\cline{1-2} \cline{4-5}
	\multicolumn{2}{c}{\revisorC} & & \multicolumn{2}{c}{\revisorD} \\
	\multicolumn{2}{c}{Revisor}   & & \multicolumn{2}{c}{Revisor}   \\[2mm]
	& \multicolumn{3}{c}{Vo. Bo.} & \\[14mm]
	\cline{2-4}
	& \multicolumn{3}{c}{\vobo}   & \\
	& \multicolumn{3}{c}{\depg}   & \\ &&&&
\end{tabular}
}\else{%%%
\begin{tabular}{p{37mm}p{21mm}p{12mm}p{21mm}p{37mm}}
	\cline{2-4}
	& \multicolumn{3}{c}{\asesor} & \\
	& \multicolumn{3}{c}{Asesor}  & \\[19mm]
	\cline{1-2} \cline{4-5}
	\multicolumn{2}{c}{\revisorA} & & \multicolumn{2}{c}{\revisorB} \\
	\multicolumn{2}{c}{Revisor}   & & \multicolumn{2}{c}{Revisor}   \\[2mm]
	& \multicolumn{3}{c}{Vo. Bo.} & \\[17mm]
	\cline{2-4}
	& \multicolumn{3}{c}{\vobo}   & \\
	& \multicolumn{3}{c}{\depg}   & \\ &&&&
\end{tabular}
}\fi%%%

\vfill

\snnl, \MakeLowercase{\fecha}

\end{center}

%%%\include{Dedicatoria}

\tableofcontents
\listoffigures
\listoftables

%Agradecimientos

\chapter{Agradecimientos}
\markboth{Agradecimientos}{}

--------(WORK IN PROGRESS)--------
%Resumen

\chapter{Resumen}
\markboth{Resumen}{}

{\setlength{\leftskip}{10mm}
\setlength{\parindent}{-10mm}

\autor.

Candidato para obtener el grado de \grado\orientacion.

\uanl.

\fime.

Título del estudio: \textsc{\titulo}.

\noindent Número de páginas: \pageref*{lastpage}.}

%%% Comienza a llenar aquí
\paragraph{Objetivos y Método de Estudio:}
En esta tesis se propone generar un software conversacional que interprete el texto introducido por un usuario y determinar su estado de ánimo, y reaccione de acuerdo con éste por medio de frases predeterminadas.

El método de estudio utilizado hará un análisis comprensivo de las redes neuronales, así como también de reconocimiento de patrones y recopilación de datos que nos dé comprensión moderada de algo tan voluble y a veces impredecible como lo es la mente humana.
\newpage
\paragraph{Contribuciones y Conclusiones:}
El algoritmo de entrenamiento utiliza un conjunto de datos específico para predecir, dentro de lo posible, qué está sintiendo una persona al momento de escribir alguna oración o frase. El algoritmo es open-source por lo que cualquier persona puede añadir o quitar módulos según se requiera.

La conclusión de esta tesis es que es posible que un algoritmo de Red Neuronal Recurrente reconozca los patrones de una frase y reaccione de acuerdo a ellas, pero se tiene que tener un conjunto de datos limpio, bien distribuido y que incluya palabras que sean difíciles de clasificar erróneamente.

\bigskip\noindent\begin{tabular}{lc}
\vspace*{-2mm}\hspace*{-2mm}Firma de la asesora: & \\
\cline{2-2} & \hspace*{1em}\asesor\hspace*{1em}
\end{tabular}




\mainmatter
\pagestyle{fime}

%%% Haz un documento para cada capítulo
\chapter{Introduction}
\label{ch1}
Human beings are social beings, this is widely known. To survive, we must band together and communicate with each other, bonding in the process. This is thanks to a neural process called \textit{empathy}, which is defined as a three-part process that happens in our brains \citep{rf1}. That happens roughly like this:
\begin{itemize}
	\item Emotional simulation centered in the limbic system, which makes us mirror the emotional elements we're watching.
	\item Processing the perspective in the prefrontal and temporal cortex.
	\item Assessing the course of action to take, either showing compassion or doing something else. This is assumed to be based in the obitofrontal cortex, as well as several other parts of the brain.	
\end{itemize}
\begin{figure}[!h]
	\centering
	\includegraphics[scale=0.3]{BrainMap}
	\caption[Lateral brain map of the parts in charge of empathy processes.]{Lateral brain map of the parts in charge of the empathy processes. Gray and pink are parts of the limbic system, cyan is part of the prefrontal cortex, green is part of the temporal cortex, blue is part of the orbitofrontal cortex. Drawing generated using BrainPainter \citep{img1}.}
	\label{fig:brainmap}
\end{figure}
This is clearly what is usually considered a human-only behavior, but there are studies that indicate that apes, dogs and rodents have been observed to take action at the presence of distress signals, either from humans or other members of their own species \citep{rf2}.
If this is true, theoretically, a machine could be taught to process signals of distress and react accordingly using a learning algorithm.

\section{Justification}
At first, the objective was to create an algorithm that could serve as a makeshift therapy chatbot that people could use when they were confused about their own feelings, but as time has passed, a lot of things have happened in my life regarding people with close-to-none empathy.
This project could prove especially useful towards being used in projects designed for people who have trouble discerning when to console someone or having an idea of how other people or even themselves feel, such as the case of people with Asperger's Syndrome or other forms of high-functioning autism.
To this end, the decision was made to work on this project.

\section{Hypothesis}
Empathy consists in a pattern of neurochemical reactions triggered by different situations. Machine learning could learn to identify these patterns. The hypothesis of this thesis is that using supervised machine learning with a neural network could accurately classify the sentiment behind an input text as ``Good'', ``Neutral'' or ``Bad'', with the purpose of being implemented in tandem with another software or algorithms focused on conversational data.

\section{Objectives}
In this section, the objectives proposed for this thesis are established.

\subsection{General Objectives}
The objective of this project is to make software capable of determining how the person that writes the input text is feeling according to the words in it, while keeping the code open-source so it can be used in other projects. This could be achieved thanks to the technology present in machine learning algorithms and an extensive amount of datasets.

\subsection{Specific Objectives}
\begin{itemize}
	\item Generating an algorithm capable of detecting key words related to mood in text.
	\item Successfully classifying the sentiment according to the input given.
	\item Making the software open-source so it can be used in other projects.
\end{itemize}

\section{Metodology}
The tools that are used in this thesis are mostly Python-based, such as TensorFlow, a neural network framework. This, combined with natural language processing tools and several filtering techniques will be used to achieve -- or at least approach as close as possible to -- the expected results while keeping the code reusable.

\section{Structure}
The content in this thesis is divided in several chapters, each one of them talking about different information about either the topics that are relevant to the scope of this project or the general process that has happened to reach the goal.\\
In the second chapter, relevant concepts are discussed and expanded upon for better understanding of what this project's purpose.
In the third chapter, existing literature is analyzed and compared to the present work, with comprehensive information and related concepts applied to each one of them.\\
In the fourth chapter, the project's design, inner workings, and the tools used are described.
In the fifth chapter, the inputs and outputs are described, and several experiments are conducted and analyzed for a better understanding of the role of every component of the algorithm.
\clearpage
\chapter{Background}
Technology in the past decades has been advancing exponentially. So much, in fact, that we can relegate data analysis to them for better accuracy and reliability than what a human can possibly achieve. This is what is called as Machine Learning (sometimes referred only as ML)
There is a variety of scenarios where it comes in handy, such as pattern recognition, which relates extensively to most of this project's work.\\
In this chapter, some key concepts will be explained for easier comprehension of this thesis and the project itself as a whole.

\section{Machine Learning}
Machine learning can be described, broadly and figuratively speaking, as a black box where some data is inserted as an input and numbers come out of it as an output \citep{rf8}.
Some more advanced models of ML allow some internal parameters inside this figurative black box to be able to be tampered with, so that some characteristics of the input data can have effect on the output, these parameters are called \textit{weights} \citep{rf9}.
Most ML algorithms have two stages: training and validation:
\begin{itemize}
\item Training processes the inputs and makes educated guesses, and in case of guessing incorrectly, depending on the obtained result, the weights are changed accordingly.
\item Validation is as simple as it sounds, some input is fed to the algorithm and information needs to be compared to the real results to test the accuracy percentage.
\end{itemize}
One of these models that is one of the most used nowadays is the one called \textit{Neural Network}.

\subsection{Neural Network}
A neural network works by using \textit{neurons}, they utilize layers that individually weigh the input given to them from the initial text or, if this has been processed already, from another neuron \citep{rf9}.
Likewise, similar to how biological brains work, these algorithms can only predict reliably if given enough data to train and validate their outputs with.

\section{Sentiment Analysis}
Sentiment Analysis (or Opinion Mining, as it is also known) as a tool for data analysis is arguably a recent happening. The term was coined in 2003 and has evolved ever since \citep{rf3}.
This type of data analysis has a lot of potential usages that have yet to be implemented in the daily life.

\subsection{Concept}
The specific execution of the algorithm varies depending on the intended purpose, but the concept and process that is used is generally the same:
\begin{itemize}
	\item The sentence to analyze is broken down to its component parts, this process is called \textit{tokenization}, and the resulting products are called, fittingly, \textit{tokens}.
	\item Every token is then tagged, making it part of an internal dictionary or \textit{lexicon}
	\item A score is assigned to every token depending on the used dataset.
\end{itemize}
The end score could be left as-is or can be reintroduced to the algorithm for a multi-layered approach depending on its focus. \citep{rf4}

\subsection{Tokenizing}
Tokenizing is the process that happens while making tokens, the way it works is very straightforward: every word in the lexicon that a machine can read is assigned a number for easier reading. Taking the following example:
\begin{center}
\fbox{This is an example text}
\end{center}

We can tell there are 6 words in the example phrase. So the tokenizing process would make the example look in the following way:
\begin{center}
\fbox{1,	2,	3,	4,	5,	6}
\end{center}

Where 1 corresponds to the word ``This'', 2 corresponds to ``is'', 3 to ``an'' and so on.

The interesting part about this process would happen if we used another example phrase, like the following:
\begin{center}
\fbox{This is another example}
\end{center}

If we did the tokenization process, it would be processed in this way:
\begin{center}
\fbox{1,	2,	7,	4}
\end{center}

Since the internal lexicon already knows some of the words in this second example, it reuses their token, adding new ones (in this example, ``another'' is 7) if needed.\\

This is fairly useful for a machine learning algorithm, since it will not have to compare such massive amount of characters in a string each time, and it would only need to evaluate integers. Whether its focus is either frequency or comparison.

\clearpage
\chapter{Related Work}
The problem proposed in this thesis is not something new by a long stretch, since sentiment analysis was developed for this very purpose. There are many applications that already apply this kind of Machine Learning for several purposes. In this chapter, some related projects are listed and analyzed.

\section{Related Projects}
\subsection{Similar Approaches}
\citet{rf10} talk about three different text classificators with a focus on sentiment analysis from Twitter: 
\begin{itemize}
\item Twitter Sentiment, which uses a Maximum Entropy algorithm.
\item Tweet Sentiments, which uses Support Vector Machines for classifications.
\item Lingpipe, which uses both previous algorithms and also Naive Bayes.
\end{itemize}
\citet{rf6} mention Koko, which uses the OpenAI API which is a counseling app for distressed teenagers. It is important to mention GPT-3 as a whole as well, which, to date, being one of the most impressive AI algorithm to be developed, the downsides being that, still in beta phase, is super resource-heavy, and its access is reserved to businesses through a fee, very expensive to use for the general public, especially students as myself. That is why in this project, TensorFlow is used, which is free to use, does not need a lot of resources to work and has the advantage of being portable once trained.
\citet{rf14} propose a chatbot developed to comprehend instructions, classifying them internally with a predefined bank of words, and reacting accordingly.

\subsection{Sentiment Analysis in Other Areas}
\citet{rf5} draft out a movie review algorithm that was capable of detecting if the review was either positive or negative depending on the words used.\\
\citet{rf11} write about a classification system to detect if a tweet was deemed as extremist or non-extremist depending on the vocabulary used and a deep-learning algorithm.\\
\citet{rf12} propose an algorithm that correlated the air pollution levels with the sentiment expressed in people's tweets.\\
\citet{rf13} mention a hierarchical attention network to detect the polarity of a customer's review, with the added bonus of being capable of learning from new data.
\citet{rf15} propose an algorithm that can detect hate speech in text using natural language text classification across several topics.
\citet{rf16} report a Recurrent Neural Network that detect political statements in YouTube comments while also classifying them in \textit{positive}, \textit{negative} or \textit{other} depending on the topic.

\section{Comparative Analysis}
Since the projects included in this chapter are all focused in the same branch of algorithm, they have some concepts in common with each other and, in turn, with this project. Some of them are:
\begin{description}
	\item[Machine Learning]{The type of algorithm needed for automatic processing, making the machine ``learn'' (hence the name) over time given enough data.}
	\item[Neural Network]{A Machine Learning algorithm that uses weights and filters to output data.}
	\item[Weights]{In ML, this is the name given to the internal value that a specific input has after being analyzed by the algorithm. With this, data classification can be achieved.}
	\item[Text Processing]{Any type of algorithm that can understand text and output data based on its contents.}
	\item[Natural Language Processing]{This is the method used for the algorithm to understand the content of the sentences, this is usually achieved by using tokenization but a preset corpus can also be used.}	
	\item[Sentiment Analysis]{This involves a ML algorithm, usually a Neural Network, that is able to analyze sentences and classify them according to the words used.}
	\item[Corpus]{Preset internal dictionary that the algorithm uses.}
	\item[Chatbot]{An algorithm that is able to reply to a prompt using natural language.}
\end{description}

\subsection{Opportunities for Improvement}
As with any project, there is always room for improvement, so in this section the areas that can be improved in them can be visualized in the following table.\\
One of the main positives of working with TensorFlow is the fact that it is a highly reusable code that can very much be ported to any system that can run Python.
\begin{table}[b!]
	\caption{Comparison between the previously discussed works and this thesis'. \checkmark indicates it fulfills that specific criteria, otherwhise $\times$ is used.}
	\vspace{0.5cm}
	\centering
	\begin{tabular}[t]{|l|l|l|l|l|l|}
	\hline
		\textbf{Project} & \rotatebox{90}{\textbf{Neural Network}} & \rotatebox{90}{\textbf{Text Processing}} & \rotatebox{90}{\textbf{Sentiment Analysis }} & \rotatebox{90}{\textbf{Chatbot}} & \rotatebox{90}{\textbf{Open Source}}
	\\ \hline
	\citet{rf10} Maximum Entropy & \checkmark & \checkmark & \checkmark & $\times$ & $\times$
	\\ \hline
	\citet{rf10} Support Vector Machines & \checkmark & \checkmark & \checkmark & $\times$ & $\times$
	\\ \hline
	\citet{rf10} Lingpipe & \checkmark & \checkmark & \checkmark & $\times$ & $\times$
	\\ \hline
	\citet{rf6} & \checkmark & \checkmark & \checkmark & \checkmark & $\times$
	\\ \hline
	\citet{rf14} & \checkmark & \checkmark & $\times$ & \checkmark & \checkmark
	\\ \hline
	\citet{rf5} & \checkmark & \checkmark & \checkmark & $\times$ & \checkmark
	\\ \hline
	\citet{rf11} & \checkmark & \checkmark & $\times$ & $\times$ & \checkmark
	\\ \hline
	\citet{rf12} & \checkmark & \checkmark & \checkmark & $\times$ & \checkmark
	\\ \hline
	\citet{rf13} & \checkmark & \checkmark & \checkmark & $\times$ & $\times$
	\\ \hline
	\citet{rf15} & \checkmark & \checkmark & $\times$ & $\times$ & \checkmark
	\\ \hline
	\citet{rf16} & \checkmark & \checkmark & \checkmark & $\times$ & \checkmark
	\\ \hline
	This thesis' project & \checkmark & \checkmark & \checkmark & \checkmark & \checkmark
	\\ \hline
	\end{tabular}
\end{table}


\clearpage
\chapter{Project Design}

\section{Datasets}



\chapter{Data Experiments}
In this chapter, the parts that compose this project as well as their context are shown. Also, some experiments are demonstrated for comparison with different parameters that could affect this thesis' project's overall accuracy.

\section{Inputs and Outputs}
In previous chapters, it has been specified that this algorithm takes an text input and, according to its contents, a message is shown as an output. The breakdown is as follows.
\subsection{Inputs}
The input that is given is cleaned up and tokenized -- as shown in the previous chapter --, , this is then added to an internal corpus that has weights set for every word in it, effectively working as scores. Every word has a different score in every label, whether it is positive or negative. This score is added up and the highest final score will be the one that the algorithm will detect as the most probable for the text input. However, this has its caveats, small sentences are more likely to be miscategorized because one word can have different applications in the scope of this project, for a more accurate analysis a longer sentence must be written.
\subsection{Outputs}
Depending on the final score, the algorithm will choose a random sentence related to the detected sentiment, this is, as of the time of writing, very rudimentary, but the fact that it is built in Python this can be a building block for a more robust, context-conscious, reply system.
In the training module, however, four extra values are part of the output as well: \textit{Loss}, \textit{Val\_loss}, \textit{Accuracy} and \textit{Val\_accuracy}.
These values are standard in every Neural Network algorithm to observe how poorly the evaluation does within the training dataset, and what the accuracy percentage is, respectively.
The \textit{Val\_} counterparts of these values are the same, but applied to the validation dataset.

\section{Experiments}
In this section, various experiments of this project are shown with varying training data and parameters with the respective accuracy and loss graphs.
The purpose of these experiments is to determine if the parameters chosen for this project are optimal and, if not, correct them and know the reason behind the improvement.
The parameters that could potentially have a great impact on the output of the classification -- and therefore are the best to experiment with -- are the following:
\begin{itemize}
	\item Used datasets: This could influentiate the amount of words in the corpus and have a big impact on how some words are percieved
	\item Training epochs: How many loops does the algorithm go through before being considered fully trained, if this number is too high it could result in \textit{overfitting}, which is, in casual terms, the Neural Network equivalent of overthinking.
	\item Units in the LSTM layer: This unit system, albeit small in the overall scale of things, could make-or-break the algoritm if not tuned correctly.
	\item Categorized sentiments: Reducing the scope of the project could potentially benefit the overall accuracy of the remaining sentiments.
\end{itemize}
The amount of improvement with each experiment is shown with loss and accuracy graphs, which are evaluated every epoch the algorithm is trained. Lower loss and higher accuracy are preferred.
\subsection{Experiment 1}
In this experiment, we look at the base version of this thesis' project.
\begin{table}[!th]
	\caption{Experiment 1's defining characteristics.}
	\vspace{0.5cm}
	\centering
	\begin{tabular}[t]{|l|l|}
	\hline
		Datasets Used & 2: \citet{d1} and \citet{d2}
	\\ \hline
		Epochs & 10
	\\ \hline
		LSTM Layer & 32 units
	\\ \hline
		Categorized Sentiments as ``Bad'' & ``Sadness'', ``Worry'', and ``Fear''.
	\\ \hline	
		 Categorized Sentiments as ``Neutral'' & ``Neutral'' and ``Boredom''.
	\\ \hline	
		Categorized Sentiments as ``Good'' & ``Happiness'', ``Fun'', ``Joy'', and ``Love''.
	\\ \hline
	\end{tabular}
\end{table}

\begin{table}[!bh]
	\caption{Experiment 1's results.}
	\vspace{0.5cm}
	\centering
	\begin{tabular}[t]{|l|l|l|l|}
	\hline
		Loss & 0.6857 & Val\_loss & 0.8499
	\\ \hline
		Accuracy & 0.7151 & Val\_accuracy & 0.6308
	\\ \hline
	\end{tabular}
\end{table}


\begin{figure}[!h]
	\centering
	\includegraphics[scale=0.8]{Accuracy_Exp1}
	\caption{Accuracy Graph of Experiment 1}
	\label{fig:accuracy_exp1}
	\includegraphics[scale=0.8]{Loss_Exp1}
	\caption{Loss Graph of Experiment 1}
	\label{fig:loss_exp1}
\end{figure}
\pagebreak

\subsection{Experiment 2}
This experiment takes sentences from one more dataset and 3 less categorized sentiments: ``Fear'', ``Joy'', and ``Love''.
\begin{table}[!th]
	\caption{Experiment 2's defining characteristics.}
	\vspace{0.5cm}
	\centering
	\begin{tabular}[t]{|l|l|}
	\hline
		Datasets Used & \makecell{3: \citet{d1}, \citet{d2} and\\ \citet{d3}}
	\\ \hline
		Epochs & 10
	\\ \hline
		LSTM Layer & 32 units
	\\ \hline
		Categorized Sentiments as ``Bad'' & ``Sadness'', and ``Worry''.
	\\ \hline	
		 Categorized Sentiments as ``Neutral'' & ``Neutral'' and ``Boredom''.
	\\ \hline	
		Categorized Sentiments as ``Good'' & ``Happiness'', and ``Fun''.
	\\ \hline
	\end{tabular}
\end{table}

\begin{table}[!bh]
	\caption{Experiment 2's results.}
	\vspace{0.5cm}
	\centering
	\begin{tabular}[t]{|l|l|l|l|}
	\hline
		Loss & 0.5956 & Val\_loss & 0.7649
	\\ \hline
		Accuracy & 0.7576 & Val\_accuracy & 0.6821
	\\ \hline
	\end{tabular}
\end{table}


\begin{figure}[!h]
	\centering
	\includegraphics[scale=0.8]{Accuracy_Exp2}
	\caption{Accuracy Graph of Experiment 2}
	\label{fig:accuracy_exp2}
	\includegraphics[scale=0.8]{Loss_Exp2}
	\caption{Loss Graph of Experiment 2}
	\label{fig:loss_exp2}
\end{figure}
\pagebreak

\subsection{Experiment 3}
Largely the same as Experiment 2, but the LSTM has more units to work with.
\begin{table}[!th]
	\caption{Experiment 3's defining characteristics.}
	\vspace{0.5cm}
	\centering
	\begin{tabular}[t]{|l|l|}
	\hline
		Datasets Used & \makecell{3: \citet{d1}, \citet{d2} and\\ \citet{d3}}
	\\ \hline
		Epochs & 10
	\\ \hline
		LSTM Layer & 64 units
	\\ \hline
		Categorized Sentiments as ``Bad'' & ``Sadness'', and ``Worry''.
	\\ \hline	
		 Categorized Sentiments as ``Neutral'' & ``Neutral'' and ``Boredom''.
	\\ \hline	
		Categorized Sentiments as ``Good'' & ``Happiness'', and ``Fun''.
	\\ \hline
	\end{tabular}
\end{table}

\begin{table}[!bh]
	\caption{Experiment 3's results.}
	\vspace{0.5cm}
	\centering
	\begin{tabular}[t]{|l|l|l|l|}
	\hline
		Loss & 0.5829 & Val\_loss & 0.7373
	\\ \hline
		Accuracy & 0.7564 & Val\_accuracy & 0.6780
	\\ \hline
	\end{tabular}
\end{table}


\begin{figure}[!h]
	\centering
	\includegraphics[scale=0.8]{Accuracy_Exp3}
	\caption{Accuracy Graph of Experiment 3}
	\label{fig:accuracy_exp3}
	\includegraphics[scale=0.8]{Loss_Exp3}
	\caption{Loss Graph of Experiment 3}
	\label{fig:loss_exp3}
\end{figure}
\pagebreak

\subsection{Experiment 4}
A mix between Experiment 1 and 2. Three datasets with the full sentiment categorization and LSTM with 32 units.
\begin{table}[!h]
	\caption{Experiment 4's defining characteristics.}
	\vspace{0.5cm}
	\centering
	\begin{tabular}[t]{|l|l|}
	\hline
		Datasets Used & \makecell{3: \citet{d1}, \citet{d2} and\\ \citet{d3}}
	\\ \hline
		Epochs & 10
	\\ \hline
		LSTM Layer & 32 units
	\\ \hline
		Categorized Sentiments as ``Bad'' & ``Sadness'', ``Worry'', and ``Fear''.
	\\ \hline	
		 Categorized Sentiments as ``Neutral'' & ``Neutral'' and ``Boredom''.
	\\ \hline	
		Categorized Sentiments as ``Good'' & ``Happiness'', ``Fun'', ``Joy'', and ``Love''.
	\\ \hline
	\end{tabular}
\end{table}

\begin{table}[!b]
	\caption{Experiment 4's results.}
	\vspace{0.5cm}
	\centering
	\begin{tabular}[t]{|l|l|l|l|}
	\hline
		Loss & 0.5455 & Val\_loss & 0.6704
	\\ \hline
		Accuracy & 0.7741 & Val\_accuracy & 0.7210
	\\ \hline
	\end{tabular}
\end{table}


\begin{figure}[!h]
	\centering
	\includegraphics[scale=0.8]{Accuracy_Exp4}
	\caption{Accuracy Graph of Experiment 4}
	\label{fig:accuracy_exp4}
	\includegraphics[scale=0.8]{Loss_Exp4}
	\caption{Loss Graph of Experiment 4}
	\label{fig:loss_exp4}
\end{figure}
\pagebreak

\subsection{Experiment 5}
Same as Experiment 4, but with an added dataset with only ``Sadness'' sentences.
\begin{table}[!h]
	\caption{Experiment 5's defining characteristics.}
	\vspace{0.5cm}
	\centering
	\begin{tabular}[t]{|l|l|}
	\hline
		Datasets Used & \makecell{4: \citet{d1}, \citet{d2},\\ \citet{d3}, and \citet{d4}}
	\\ \hline
		Epochs & 10
	\\ \hline
		LSTM Layer & 32 units
	\\ \hline
		Categorized Sentiments as ``Bad'' & ``Sadness'', ``Worry'', and ``Fear''.
	\\ \hline	
		 Categorized Sentiments as ``Neutral'' & ``Neutral'' and ``Boredom''.
	\\ \hline	
		Categorized Sentiments as ``Good'' & ``Happiness'', ``Fun'', ``Joy'', and ``Love''.
	\\ \hline
	\end{tabular}
\end{table}

\begin{table}[!b]
	\caption{Experiment 5's results.}
	\vspace{0.5cm}
	\centering
	\begin{tabular}[t]{|l|l|l|l|}
	\hline
		Loss & 0.5442 & Val\_loss & 0.6620
	\\ \hline
		Accuracy & 0.6620 & Val\_accuracy & 0.7162
	\\ \hline
	\end{tabular}
\end{table}


\begin{figure}[!h]
	\centering
	\includegraphics[scale=0.8]{Accuracy_Exp5}
	\caption{Accuracy Graph of Experiment 5}
	\label{fig:accuracy_exp5}
	\includegraphics[scale=0.8]{Loss_Exp5}
	\caption{Loss Graph of Experiment 5}
	\label{fig:loss_exp5}
\end{figure}
\pagebreak

\subsection{Experiment 6}
This is the largest change on an experiment, the ``Neutral'' category has been completely disabled with the purpose of seeing how the rest of the data would be classified as.
\begin{table}[!h]
	\caption{Experiment 6's defining characteristics.}
	\vspace{0.5cm}
	\centering
	\begin{tabular}[t]{|l|l|}
	\hline
		Datasets Used & \makecell{4: \citet{d1}, \citet{d2},\\ \citet{d3}, and \citet{d4}}
	\\ \hline
		Epochs & 10
	\\ \hline
		LSTM Layer & 32 units
	\\ \hline
		Categorized Sentiments as ``Bad'' & ``Sadness'', ``Worry'', and ``Fear''.
	\\ \hline	
		 Categorized Sentiments as ``Neutral'' & N/A
	\\ \hline	
		Categorized Sentiments as ``Good'' & ``Happiness'', ``Fun'', ``Joy'', and ``Love''.
	\\ \hline
	\end{tabular}
\end{table}

\begin{table}[!b]
	\caption{Experiment 6's results.}
	\vspace{0.5cm}
	\centering
	\begin{tabular}[t]{|l|l|l|l|}
	\hline
		Loss & 0.6222 & Val\_loss & 0.7186
	\\ \hline
		Accuracy & 0.6550 & Val\_accuracy & 0.5357
	\\ \hline
	\end{tabular}
\end{table}


\begin{figure}[!h]
	\centering
	\includegraphics[scale=0.8]{Accuracy_Exp6}
	\caption{Accuracy Graph of Experiment 6}
	\label{fig:accuracy_exp6}
	\includegraphics[scale=0.8]{Loss_Exp6}
	\caption{Loss Graph of Experiment 6}
	\label{fig:loss_exp6}
\end{figure}
\pagebreak

\subsection{Experiment 7}
Largely the same as Experiment 5 with half the epochs. This with the purpose of seeing if the data has been overfit.
\begin{table}[!h]
	\caption{Experiment 7's defining characteristics.}
	\vspace{0.5cm}
	\centering
	\begin{tabular}[t]{|l|l|}
	\hline
		Datasets Used & \makecell{4: \citet{d1}, \citet{d2},\\ \citet{d3}, and \citet{d4}}
	\\ \hline
		Epochs & 5
	\\ \hline
		LSTM Layer & 32 units
	\\ \hline
		Categorized Sentiments as ``Bad'' & ``Sadness'', ``Worry'', and ``Fear''.
	\\ \hline	
		 Categorized Sentiments as ``Neutral'' & ``Neutral'' and ``Boredom''.
	\\ \hline	
		Categorized Sentiments as ``Good'' & ``Happiness'', ``Fun'', ``Joy'', and ``Love''.
	\\ \hline
	\end{tabular}
\end{table}

\begin{table}[!b]
	\caption{Experiment 7's results.}
	\vspace{0.5cm}
	\centering
	\begin{tabular}[t]{|l|l|l|l|}
	\hline
		Loss & 0.6041 & Val\_loss & 0.6555
	\\ \hline
		Accuracy & 0.7451 & Val\_accuracy & 0.7197
	\\ \hline
	\end{tabular}
\end{table}


\begin{figure}[!h]
	\centering
	\includegraphics[scale=0.8]{Accuracy_Exp7}
	\caption{Accuracy Graph of Experiment 7}
	\label{fig:accuracy_exp7}
	\includegraphics[scale=0.8]{Loss_Exp7}
	\caption{Loss Graph of Experiment 7}
	\label{fig:loss_exp7}
\end{figure}
\pagebreak

\appendix
%%% Haz un documento para cada apéndice
%%%\\chapter{Este es un apéndice}

\section{Citas bibliográficas}

\section{Comillas}



\backmatter
\pagestyle{main}

%%% Aquí va la bibliografía, puedes usar el entorno de LaTeX (thebibliography)
%%% o la herramienta BibTeX. En caso de que optes por BibTeX, puedes usar
%%% alguno de los archivos de estilo (mighelbib.bst o mighelnat.bst) incluidos
%%% en el paquete, cuyos diseños armonizan con el diseño de tesis provisto por
%%% fime.cls. Para muestra, basta un botón:
\bibliographystyle{plainnat}
\bibliography{biblio}

\label{lastpage}
%Autobiografia

\chapter*{Resumen autobiográfico}
\thispagestyle{empty}

\begin{center}
\autor

Candidato para obtener el grado de\\
\grado\\
\orientacion\bigskip

\uanl\\
\fime\bigskip

Tesis:\\
\textsc{\large\titulo}
\end{center}\bigskip

%Aquí va tu historia
Nací el 17 de Noviembre de 1998 en Monterrey, Nuevo León, el mayor de los hijos de José Artemio Espronceda Estrada y Yadhira Lizet Gómez García. Me apasiona mucho el área de Análisis de Datos y Aprendizaje Máquina (Machine Learning), así como áreas como el Diseño de Videojuegos y la Psicología, por lo que este proyecto es la culminación entre mis pasiones más grandes para concluir la carrera de Ingeniería de Tecnología de Software.

\end{document}
