%Resumen

\chapter{Resumen}
\markboth{Resumen}{}

{\setlength{\leftskip}{10mm}
\setlength{\parindent}{-10mm}

\autor.

Candidato para obtener el grado de \grado\orientacion.

\uanl.

\fime.

Título del estudio: \textsc{\titulo}.

\noindent Número de páginas: \pageref*{lastpage}.}

%%% Comienza a llenar aquí
\paragraph{Objetivos y Método de Estudio:}
En esta tesis se propone generar software de uso libre que interprete el texto introducido por un usuario y determine su estado de ánimo, con el propósito de usarse en conjunto con otro software o algoritmos con enfoque en datos conversacionales.

El método de estudio utilizado hará un análisis comprensivo de las redes neuronales y aprendizaje máquina supervisado, así como también de reconocimiento de patrones y recopilación de datos que nos dé comprensión moderada de algo tan voluble y a veces impredecible como lo es la mente humana.
\newpage
\paragraph{Contribuciones y Conclusiones:}
El algoritmo de entrenamiento utiliza un conjunto de datos específico para predecir, dentro de lo posible, qué está sintiendo una persona al momento de escribir alguna oración o frase. El algoritmo resultante es open-source por lo que cualquier persona puede añadir o quitar módulos según se requiera, y utilizarlo en otros proyectos.

La conclusión de esta tesis es que es posible que un algoritmo de Red Neuronal Recurrente reconozca los patrones de una frase usando código open-source, pero se tiene que tener un conjunto de datos limpio, bien distribuido y que incluya palabras que sean difíciles de clasificar erróneamente.

\bigskip\noindent\begin{tabular}{lc}
\vspace*{-2mm}\hspace*{-2mm}Firma de la asesora: & \\
\cline{2-2} & \hspace*{1em}\asesor\hspace*{1em}
\end{tabular}


