\chapter{Related Work}
The problem proposed in this thesis is not something new by a long stretch, since sentiment analysis was developed for this very purpose. There are many applications that already apply this kind of Machine Learning for several purposes. In this chapter, some related projects are listed and analyzed.

\section{Related Projects}
\subsection{Similar Approaches}
\citet{rf10} talk about three different text classificators with a focus on sentiment analysis from Twitter: 
\begin{itemize}
\item Twitter Sentiment, which uses a Maximum Entropy algorithm.
\item Tweet Sentiments, which uses Support Vector Machines for classifications.
\item Lingpipe, which uses both previous algorithms and also Naive Bayes.
\end{itemize}
\citet{rf6} mentions Koko, which uses the OpenAI API which is a counseling app for distressed teenagers. It's important to mention GPT-3 as a whole as well, which, to date, it's one of the most impressive AI algorithm to be developed, the downsides being that it's still in beta phase, it's super resource-heavy, and its access is reserved to businesses through a fee, very expensive to use for the general public, especially students as myself. That's why in this project, TensorFlow is used, which is free to use, doesn't need a lot of resources to work and it's portable once it's trained.
\citet{rf14} propose a chatbot developed to comprehend instructions, classifying them internally with a predefined bank of words, and reacting accordingly.

\subsection{Sentiment Analysis in other areas}
\citet{rf5} drafted out a movie review algorithm that was capable of detecting if the review was either positive or negative depending on the words used.\\
\citet{rf11} developed a classification system to detect if a tweet was deemed as extremist or non-extremist depending on the vocabulary used and a deep-learning algorithm.\\
\citet{rf12} proposed an algorithm that correlated the air pollution levels with the sentiment expressed in people's tweets.\\
\citet{rf13} mentions a hierarchical attention network to detect the polarity of a customer's review, with the added bonus of being capable of learning from new data.
\citet{rf15} developed an algorithm that can detect hate speech in text using natural language text classification across several topics.
\citet{rf16} report a Recurrent Neural Network that detect political statements in YouTube comments while also classifying them in \textit{positive}, \textit{negative} or \textit{other} depending on the topic.



\clearpage