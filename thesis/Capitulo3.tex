\chapter{Related Work}
The problem proposed in this thesis is not something new by a long stretch, since sentiment analysis was developed for this very purpose. There are many applications that already apply this kind of Machine Learning for several purposes. In this chapter, some related projects are listed and analyzed.

\section{Related Projects}
In this section, some literature is listed which proposes projects which have similar approaches to the present work, and some others that may not have the same objectives in mind but use algorithms that could be applied as well.
\subsection{Similar Approaches}
\citet{rf10} talk about three different text classificators with a focus on sentiment analysis from Twitter:\\ 
\begin{itemize}
\item Twitter Sentiment, which uses a Maximum Entropy algorithm\footnote{This algorithm works by having the bias that certain characteristics repeat more in certain categories in text. If no bias is found, the distribution is uniform. \citep{rf17}}.
\item Tweet Sentiments, which uses Support Vector Machines\footnote{Binary algorithm that can sort between two classes, or opt for classification in a ``one-versus-everything else'' basis. \citep{rf18}} for classifications.
\item Lingpipe, which uses both previous algorithms and also Naive Bayes\footnote{This algorithm utilizes weights expressed in \textit{-1}, \textit{0}, or \textit{+1} depending on the sensitivity of specific characteristics \citep{rf19}. Works very similarly to a classic perceptron, which only uses\\ \textit{0} or \textit{1}.}
\end{itemize}
\citet{rf6} mention Koko, which uses the OpenAI API which is a counseling app for distressed teenagers in need of immediate psychological support, composed of a chatbot and sentiment analysis capabilities.\\
\citet{rf14} propose a chatbot developed to comprehend instructions, classifying them internally with a predefined bank of words, and reacting accordingly.

\subsection{Sentiment Analysis in Other Areas}
\citet{rf5} draft out a movie review algorithm that was capable of detecting if the review was either positive or negative depending on the words used.\\
\citet{rf11} write about a classification system to detect if a tweet was deemed as extremist or non-extremist depending on the vocabulary used and a deep-learning algorithm.\\
\citet{rf12} propose an algorithm that correlated the air pollution levels with the sentiment expressed in people's tweets.\\
\citet{rf13} mention a hierarchical attention network to detect the polarity of a customer's review, with the added bonus of being capable of learning from new data.
\citet{rf15} propose an algorithm that can detect hate speech in text using natural language text classification across several topics.
\citet{rf16} report a Recurrent Neural Network that detect political statements in YouTube comments while also classifying them in \textit{positive}, \textit{negative}, or \textit{other} depending on the topic.

\section{Comparative Analysis}
Since the projects included in this chapter are all focused in the same branch of algorithm, they have some concepts in common with each other and, in turn, with this project. Some of them are:
\begin{description}
	\item[Machine Learning]{The type of algorithm needed for automatic processing, making the machine ``learn'' (hence the name) over time given enough data.}
	\item[Neural Network]{A Machine Learning algorithm that uses weights and filters to output data.}
	\item[Weights]{In ML, this is the name given to the internal value that a specific input has after being analyzed by the algorithm. With this, data classification can be achieved.}
	\item[Text Processing]{Any type of algorithm that can understand text and output data based on its contents.}
	\item[Natural Language Processing]{This is the method used for the algorithm to understand the content of the sentences, this is usually achieved by using tokenization but a preset corpus can also be used.}	
	\item[Sentiment Analysis]{This involves a ML algorithm, usually a Neural Network, that is able to analyze sentences and classify them according to the words used.}
	\item[Corpus]{Preset internal dictionary that the algorithm uses.}
	\item[Chatbot]{An algorithm that is able to reply to a prompt using natural language.}
\end{description}

\subsection{Opportunities for Improvement}
One of the main positives of working with TensorFlow is the fact that it is a highly reusable code that can very much be ported to any system that can run Python.\\
It is important to mention GPT-3 as a whole, the framework that Koko -- mentioned by \citet{rf6} -- uses is, to date, one of the most impressive AI algorithm to be developed, the downsides being that, being still in beta phase, is very resource-heavy, and its access is reserved to businesses through a fee, very expensive to use for the general public, especially students. That is why in this project, TensorFlow is used, which is free to use, does not need a lot of resources to work and has the advantages of being portable once trained, and also being easily modifiable if needed.
\begin{table}[h!]
	\caption{Comparison between existing literature and the present work. \checkmark indicates the fulfillment of a criterion, otherwise $\times$ is used.}
	\vspace{0.5cm}
	\centering
	\begin{tabular}[t]{|l|l|l|l|l|l|}
	\hline
		\textbf{Project} & \rotatebox{90}{\textbf{Neural Network}} & \rotatebox{90}{\textbf{Text Processing}} & \rotatebox{90}{\textbf{Sentiment Analysis }} & \rotatebox{90}{\textbf{Chatbot}} & \rotatebox{90}{\textbf{Open Source}}
	\\ \hline
	\citet{rf10} Maximum Entropy & \checkmark & \checkmark & \checkmark & $\times$ & $\times$
	\\ \hline
	\citet{rf10} Support Vector Machines & \checkmark & \checkmark & \checkmark & $\times$ & $\times$
	\\ \hline
	\citet{rf10} Lingpipe & \checkmark & \checkmark & \checkmark & $\times$ & $\times$
	\\ \hline
	\citet{rf6} & \checkmark & \checkmark & \checkmark & \checkmark & $\times$
	\\ \hline
	\citet{rf14} & \checkmark & \checkmark & $\times$ & \checkmark & \checkmark
	\\ \hline
	\citet{rf5} & \checkmark & \checkmark & \checkmark & $\times$ & \checkmark
	\\ \hline
	\citet{rf11} & \checkmark & \checkmark & $\times$ & $\times$ & \checkmark
	\\ \hline
	\citet{rf12} & \checkmark & \checkmark & \checkmark & $\times$ & \checkmark
	\\ \hline
	\citet{rf13} & \checkmark & \checkmark & \checkmark & $\times$ & $\times$
	\\ \hline
	\citet{rf15} & \checkmark & \checkmark & $\times$ & $\times$ & \checkmark
	\\ \hline
	\citet{rf16} & \checkmark & \checkmark & \checkmark & $\times$ & \checkmark
	\\ \hline
	The present work & \checkmark & \checkmark & \checkmark & \checkmark & \checkmark
	\\ \hline
	\end{tabular}
\end{table}


\clearpage