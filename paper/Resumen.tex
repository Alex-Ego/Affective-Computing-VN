%Resumen

\chapter{Resumen}
\markboth{Resumen}{}

{\setlength{\leftskip}{10mm}
\setlength{\parindent}{-10mm}

\autor.

Candidato para obtener el grado de \grado\orientacion.

\uanl.

\fime.

Título del estudio: \textsc{\titulo}.

\noindent Número de páginas: \pageref*{lastpage}.}

%%% Comienza a llenar aquí
\paragraph{Objetivos y método de estudio:}
En esta tesis se propone generar un software conversacional que interprete el texto introducido por un usuario y determinar su estado de ánimo, y reaccione de acuerdo con éste por medio de frases predeterminadas.

El método de estudio utilizado hará un análisis comprensivo de las redes neuronales, así como también de la comprensión suficiente de algo tan voluble y a veces impredecible como lo es la mente humana.
\newpage
\paragraph{Contribuciones y conlusiones:}
El algoritmo de entrenamiento utiliza un conjunto de datos específico para predecir lo más acertadamente posible qué está sintiendo una persona al momento de escribir alguna oración o frase.

\bigskip\noindent\begin{tabular}{lc}
\vspace*{-2mm}\hspace*{-2mm}Firma del asesor: & \\
\cline{2-2} & \hspace*{1em}\asesor\hspace*{1em}
\end{tabular}


