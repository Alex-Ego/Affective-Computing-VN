%Resumen

\chapter{Resumen}
\markboth{Resumen}{}

{\setlength{\leftskip}{10mm}
\setlength{\parindent}{-10mm}

\autor.

Candidato para obtener el grado de \grado\orientacion.

\uanl.

\fime.

Título del estudio: \textsc{\titulo}.

\noindent Número de páginas: \pageref*{lastpage}.}

%%% Comienza a llenar aquí
\paragraph{Objetivos y método de estudio:}
Cuantificar el inventario forestal por medio de un dron y el aprendizaje máquina con el fin de identificar cada tipo de especie de arbol en diferentes instancias de acuerdo a un recorrido. El sitio recorrido por el dron, es la zona del Cilantrillo, úbicado en Santiago, Nuevo León, México.).

\paragraph{Contribuciones y conlusiones:}
El algoritmo de entrenamiento capaz de identificar las especies de arboles mediante un entrenamiento, mismo que se genero a través de instancias fotográficas.

\bigskip\noindent\begin{tabular}{lc}
\vspace*{-2mm}\hspace*{-2mm}Firma del asesor: & \\
\cline{2-2} & \hspace*{1em}\asesor\hspace*{1em}
\end{tabular}


