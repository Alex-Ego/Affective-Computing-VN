%%%%%%%%%%%%%%%%%%%%%
% Documento maestro %
%%%%%%%%%%%%%%%%%%%%%
\documentclass{fime}

%%%%%%%%%%%%%%%%%%%%%%%%%%%%%%%%%%%%%%%%%%%
% Cargando paquetes y definiendo opciones %
%%%%%%%%%%%%%%%%%%%%%%%%%%%%%%%%%%%%%%%%%%%
% Aquí puedes cargar los paquetes que vas a usar. La clase
% fime ya incluye babel, inputenc, graphicx y los de la AMS.
% Cargar un paquete está a tu libertad (y responsabilidad).
\usepackage{hyperref}
\usepackage{graphicx}
\usepackage{pdflscape}
\usepackage{amsmath}
\usepackage[utf8]{inputenc}
\usepackage{natbib}
\usepackage[bottom]{footmisc}
\usepackage{float}
\usepackage{footmisc}
\usepackage{adjustbox}
\usepackage{tikz}

\setlength{\footnotesep}{0.5cm}
\setlength\footnotemargin{3pt}

    \hypersetup{breaklinks=true,colorlinks=true,
        linkcolor=black,citecolor=black,urlcolor=black}

%%%%%%%%%%%%%%%%%%%%%
% Definiendo campos %
%%%%%%%%%%%%%%%%%%%%%
\def\titulo{Sentiment Analysis through a chatbot}
\def\autor{Alexander Espronceda Gómez}
\def\matricula{1742000}
\def\grado{Ingeniería en Tecnología de Software}
% En caso de que el grado tenga orientación o especialidad llenar el siguiente
% campo dejando un ESPACIO INICIAL, en caso contrario, dejar vacío
\def\orientacion{}
\def\fecha{Agosto 2021} % Coloca el mes con mayúscula inicial

\def\asesor{Dra. Satu Elisa Schaeffer}
\def\revisorA{Nombre del revisor A}
\def\revisorB{Nombre del revisor B}
% En el caso de que tu tesis sea de doctorado activa la variable cambiándola a \doctoradotrue
% y define tus otros dos revisores
\newif\ifdoctorado\doctoradofalse
\def\revisorC{Nombre del revisor C}
\def\revisorD{Nombre del revisor D}
% El visto bueno siempre va
\def\vobo{Dr. ..}

%%%%%%%%%%%%%%%%%%%%%%%
% Inicia el documento %
%%%%%%%%%%%%%%%%%%%%%%%
\begin{document}

\frontmatter
\pagestyle{main}

%%% Incluye PortillasM si tu tesis es de Maestría
%%% y PortillasD si es de doctorado.
% Portadas (Maestría)

\def\uanl{Universidad Autónoma de Nuevo León}
\def\fime{Facultad de Ingeniería Mecánica y Eléctrica}
\def\depg{Subdirección de Estudios de Posgrado}
\def\snnl{San Nicolás de los Garza, Nuevo León}

%%%%%%%%%%%%%%%%%%%%%%%%
% Primer portada: UANL %
%%%%%%%%%%%%%%%%%%%%%%%%
\thispagestyle{empty}

\begin{scshape}
\begin{center}
	{\Large \uanl} \\[5mm]
	{\large \fime} \\[5mm]
	{\large \depg}
	\vskip 15mm
	\includegraphics[height=55mm]{uanl}
	\vskip 12mm
	\begin{tabular}{p{11cm}}
		\centering
		{\large \titulo}
	\end{tabular}
	\vskip 7mm
	{por}\\[7mm]
	{\large \autor}\\[7mm]
	{como requisito parcial para obtener el grado de}\\[3mm]
	\MakeUppercase{\grado}\\
	\orientacion
	\vfill
	\fecha
\end{center}
\end{scshape}

%%%%%%%%%%%%%%%%%%%%%%%%%
% Segunda portada: FIME %
%%%%%%%%%%%%%%%%%%%%%%%%%
\newpage
\thispagestyle{empty}

\begin{scshape}
\begin{center}
	{\Large \uanl} \\[5mm]
	{\large \fime} \\[5mm]
	{\large \depg}
	\vskip 16mm
	\includegraphics[height=50mm]{fime}
	\vskip 16mm
	\begin{tabular}{p{11cm}}
		\centering
		{\large \titulo}
	\end{tabular}
	\vskip 7mm
	{por}\\[7mm]
	{\large \autor}\\[7mm]
	{como requisito parcial para obtener el grado de}\\[3mm]
	\MakeUppercase{\grado}\\
	\orientacion
	\vfill
	\fecha
\end{center}
\end{scshape}

%%%%%%%%%%%%%%%%%%%%%%%%%%%%%
% Carta del comité de tesis %
%%%%%%%%%%%%%%%%%%%%%%%%%%%%%
\newpage
\thispagestyle{empty}
\enlargethispage{5mm}

\begin{center}
{\bf \large \uanl} \\
{\bf \fime} \\
{\bf \depg}
\end{center}
\vskip 4mm

Los miembros del Comité de Tesis recomendamos que la Tesis <<\titulo>>, realizada por el alumno \autor, con número de matrícula \matricula, sea aceptada para su defensa como requisito parcial para obtener el grado de \grado\orientacion.
\ifdoctorado{\vskip 10mm}\else{\vskip 8mm}\fi

\begin{center}
El Comité de Tesis\\
\ifdoctorado{\vskip 15mm}\else{\vskip 25mm}\fi

\ifdoctorado{%%%
\begin{tabular}{p{37mm}p{21mm}p{12mm}p{21mm}p{37mm}}
	\cline{2-4}
	& \multicolumn{3}{c}{\asesor} & \\
	& \multicolumn{3}{c}{Asesor}  & \\[15mm]
	\cline{1-2} \cline{4-5}
	\multicolumn{2}{c}{\revisorA} & & \multicolumn{2}{c}{\revisorB} \\
	\multicolumn{2}{c}{Revisor}   & & \multicolumn{2}{c}{Revisor}   \\[17mm]
	\cline{1-2} \cline{4-5}
	\multicolumn{2}{c}{\revisorC} & & \multicolumn{2}{c}{\revisorD} \\
	\multicolumn{2}{c}{Revisor}   & & \multicolumn{2}{c}{Revisor}   \\[2mm]
	& \multicolumn{3}{c}{Vo. Bo.} & \\[14mm]
	\cline{2-4}
	& \multicolumn{3}{c}{\vobo}   & \\
	& \multicolumn{3}{c}{\depg}   & \\ &&&&
\end{tabular}
}\else{%%%
\begin{tabular}{p{37mm}p{21mm}p{12mm}p{21mm}p{37mm}}
	\cline{2-4}
	& \multicolumn{3}{c}{\asesor} & \\
	& \multicolumn{3}{c}{Asesor}  & \\[19mm]
	\cline{1-2} \cline{4-5}
	\multicolumn{2}{c}{\revisorA} & & \multicolumn{2}{c}{\revisorB} \\
	\multicolumn{2}{c}{Revisor}   & & \multicolumn{2}{c}{Revisor}   \\[2mm]
	& \multicolumn{3}{c}{Vo. Bo.} & \\[17mm]
	\cline{2-4}
	& \multicolumn{3}{c}{\vobo}   & \\
	& \multicolumn{3}{c}{\depg}   & \\ &&&&
\end{tabular}
}\fi%%%

\vfill

\snnl, \MakeLowercase{\fecha}

\end{center}

%%%\include{Dedicatoria}

\tableofcontents
\listoffigures
\listoftables

%Agradecimientos

\chapter{Agradecimientos}
\markboth{Agradecimientos}{}

--------(WORK IN PROGRESS)--------
%Resumen

\chapter{Resumen}
\markboth{Resumen}{}

{\setlength{\leftskip}{10mm}
\setlength{\parindent}{-10mm}

\autor.

Candidato para obtener el grado de \grado\orientacion.

\uanl.

\fime.

Título del estudio: \textsc{\titulo}.

\noindent Número de páginas: \pageref*{lastpage}.}

%%% Comienza a llenar aquí
\paragraph{Objetivos y Método de Estudio:}
En esta tesis se propone generar un software conversacional que interprete el texto introducido por un usuario y determinar su estado de ánimo, y reaccione de acuerdo con éste por medio de frases predeterminadas.

El método de estudio utilizado hará un análisis comprensivo de las redes neuronales, así como también de reconocimiento de patrones y recopilación de datos que nos dé comprensión moderada de algo tan voluble y a veces impredecible como lo es la mente humana.
\newpage
\paragraph{Contribuciones y Conclusiones:}
El algoritmo de entrenamiento utiliza un conjunto de datos específico para predecir, dentro de lo posible, qué está sintiendo una persona al momento de escribir alguna oración o frase. El algoritmo es open-source por lo que cualquier persona puede añadir o quitar módulos según se requiera.

La conclusión de esta tesis es que es posible que un algoritmo de Red Neuronal Recurrente reconozca los patrones de una frase y reaccione de acuerdo a ellas, pero se tiene que tener un conjunto de datos limpio, bien distribuido y que incluya palabras que sean difíciles de clasificar erróneamente.

\bigskip\noindent\begin{tabular}{lc}
\vspace*{-2mm}\hspace*{-2mm}Firma de la asesora: & \\
\cline{2-2} & \hspace*{1em}\asesor\hspace*{1em}
\end{tabular}




\mainmatter
\pagestyle{fime}

%%% Haz un documento para cada capítulo
\chapter{Introduction}
Human beings are social beings, this is widely known. To survive, we must band together and communicate with each other, bonding in the process. This is thanks to a neural process called \textbf{empathy}, which is defined as a three-part process that happens in our brains \citep{rf1}. That roughly happens like this:
\begin{itemize}
	\item Emotional simulation centered in the limbic system, which makes us mirror the emotional elements we're watching.
	\item Processing the perspective in the prefrontal and temporal cortex.
	\item Assessing the course of action to take, either showing compassion or doing something else. This is allegedly based in the obitofrontal cortex, as well as several other parts of the brain.
\end{itemize}
This is clearly what we consider a normally human-only behavior, but there's been studies that indicate that apes, dogs and rodents have been observed to take action at the presence of distress signals, either from humans or other members of their species \citep{rf2}.
If this is true, theoretically, we could teach a machine learning algorithm to process signals of distress and react accordingly.

\section{Motivation}
At first, I wanted to create an algorithm that could serve as a makeshift therapy chatbot that people could use when they were confused about their own feelings, but as time has passed, a lot of things have happened in my life regarding people with close-to-none empathy.
Taking the previous part into consideration, it means that there's a pattern that could be learned rather than processed biologically. This could prove especially useful towards people who have trouble discerning when to console someone or having an idea of how other people feel, such as the case of people with Asperger's Syndrome or other forms of high-functioning autism.
To this end, I've decided to work on this project.
\clearpage

\appendix
%%% Haz un documento para cada apéndice
%%%\\chapter{Este es un apéndice}

\section{Citas bibliográficas}

\section{Comillas}



\backmatter
\pagestyle{main}

%%% Aquí va la bibliografía, puedes usar el entorno de LaTeX (thebibliography)
%%% o la herramienta BibTeX. En caso de que optes por BibTeX, puedes usar
%%% alguno de los archivos de estilo (mighelbib.bst o mighelnat.bst) incluidos
%%% en el paquete, cuyos diseños armonizan con el diseño de tesis provisto por
%%% fime.cls. Para muestra, basta un botón:
\bibliographystyle{mighelnat}
\bibliography{biblio}

\label{lastpage}
%Autobiografia

\chapter*{Resumen autobiográfico}
\thispagestyle{empty}

\begin{center}
\autor

Candidato para obtener el grado de\\
\grado\\
\orientacion\bigskip

\uanl\\
\fime\bigskip

Tesis:\\
\textsc{\large\titulo}
\end{center}\bigskip

%Aquí va tu historia
Nací el 17 de Noviembre de 1998 en Monterrey, Nuevo León, el mayor de los hijos de José Artemio Espronceda Estrada y Yadhira Lizet Gómez García. Me apasiona mucho el área de Análisis de Datos y Aprendizaje Máquina (Machine Learning), así como áreas como el Diseño de Videojuegos y la Psicología, por lo que este proyecto es la culminación entre mis pasiones más grandes para concluir la carrera de Ingeniería de Tecnología de Software.

\end{document}
