\chapter{Introduction}
Human beings are social beings, this is widely known. To survive, we must band together and communicate with each other, bonding in the process. This is thanks to a neural process called \textbf{empathy}, which is defined as a three-part process that happens in our brains \citep{rf1}. That roughly happens like this:
\begin{itemize}
	\item Emotional simulation centered in the limbic system, which makes us mirror the emotional elements we're watching.
	\item Processing the perspective in the prefrontal and temporal cortex.
	\item Assessing the course of action to take, either showing compassion or doing something else. This is allegedly based in the obitofrontal cortex, as well as several other parts of the brain.
\end{itemize}
This is clearly what we consider a normally human-only behavior, but there's been studies that indicate that apes, dogs and rodents have been observed to take action at the presence of distress signals, either from humans or other members of their species \citep{rf2}.
If this is true, theoretically, we could teach a machine learning algorithm to process signals of distress and react accordingly.

\section{Motivation}
At first, I wanted to create an algorithm that could serve as a makeshift therapy chatbot that people could use when they were confused about their own feelings, but as time has passed, a lot of things have happened in my life regarding people with close-to-none empathy.
Taking the previous part into consideration, it means that there's a pattern that could be learned rather than processed biologically. This could prove especially useful towards people who have trouble discerning when to console someone or having an idea of how other people feel, such as the case of people with Asperger's Syndrome or other forms of high-functioning autism.
To this end, I've decided to work on this project.
\clearpage