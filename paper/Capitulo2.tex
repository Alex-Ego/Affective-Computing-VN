\chapter{Sentiment Analysis}
Sentiment Analysis (or Opinion Mining, as it is also known) as a tool for data analysis is arguably a recent happening. The term was coined in 2003 and has evolved ever since \citep{rf3}.
This type of data analysis has a lot of potential usages that have yet to be implemented in the daily life.

\section{Concept}
The specific execution of the algorithm varies depending on the intended purpose, but the concept and process that is used is generally the same:
\begin{itemize}
	\item The sentence to analyze is broken down to its component parts, this process is called \textit{tokenization}, and the resulting products are called, fittingly, \textit{tokens}.
	\item Every token is then tagged, making it part of an internal dictionary or \textit{lexicon}
	\item A score is assigned to every token depending on the used dataset.
\end{itemize}
The end score could be left as-is or can be reintroduced to the algorithm for a multi-layered approach depending on its focus. \citep{rf4}

\section{Possible Usages}


\clearpage