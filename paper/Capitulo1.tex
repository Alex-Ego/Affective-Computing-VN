\chapter{Introduction}
Human beings are social beings, this is widely known. To survive, we must band together and communicate with each other, bonding in the process. This is thanks to a neural process called \textbf{empathy}, which is defined as a three-part process that happens in our brains \citep{rf1}. That roughly happens like this:
\begin{itemize}
	\item Emotional simulation centered in the limbic system, which makes us mirror the emotional elements we're watching.
	\item Processing the perspective in the prefrontal and temporal cortex.
	\item Assessing the course of action to take, either showing compassion or doing something else. This is allegedly based in the obitofrontal cortex, as well as several other parts of the brain.
\end{itemize}
This is clearly what it's usually considered a human-only behavior, but there's been studies that indicate that apes, dogs and rodents have been observed to take action at the presence of distress signals, either from humans or other members of their own species \citep{rf2}.
If this is true, theoretically, a machine could be taught to process signals of distress and react accordingly using a learning algorithm.

\section{Justification}
At first, the objective was to create an algorithm that could serve as a makeshift therapy chatbot that people could use when they were confused about their own feelings, but as time has passed, a lot of things have happened in my life regarding people with close-to-none empathy.
This project could prove especially useful towards people who have trouble discerning when to console someone or having an idea of how other people or even themselves feel, such as the case of people with Asperger's Syndrome or other forms of high-functioning autism.
To this end, the decision was made to work on this project.

\section{Hypothesis}
Empathy consists in a pattern of neurochemical reactions triggered by different situations. Machine learning could learn to identify these patterns without them being processed biologically.

\section{Objectives}
In this section, the objectives proposed for this paper are established.

\subsection{General Objectives}
The objective of this project is to determine how the user's feeling according to the words in the input. This could be achieved thanks to the technology present in machine learning algorithms and an extensive amount of datasets.

\subsection{Specific Objectives}
\begin{itemize}
	\item Generating an algorithm capable of detecting key words related to the user's mood.
	\item Predicting successfully the user's mood according to their input.
\end{itemize}

\section{Metodology}
The tools that are used in this paper are mostly Python-based, such as TensorFlow 2.0, a neural network framework. This, combined with natural language processing tools and several filtering techniques will be used to achieve -- or at least approach as close as possible to -- the expected results.

\section{Structure}
The content in this thesis is divided in several chapters, each one of them talking about different information about either the topics that are relevant to the scope of this project or the general process that has happened to reach the goal.\\
In the second chapter, topics like relevant concepts are discussed and expanded upon, also related content and similar projects are looked upon and compared to this project.\\
In the third chapter, a general approach to the project's process is described, with some screenshots of the relevant information.
\clearpage