\chapter{Introduction}
Human beings are social beings, this is widely known. To survive, we must band together and communicate with each other, bonding in the process. This is thanks to a neural process called \textbf{empathy}, which is defined as a three-part process that happens in our brains \citep{rf1}. That roughly happens like this:
\begin{itemize}
	\item Emotional simulation centered in the limbic system, which makes us mirror the emotional elements we're watching.
	\item Processing the perspective in the prefrontal and temporal cortex.
	\item Assessing the course of action to take, either showing compassion or doing something else. This is allegedly based in the obitofrontal cortex, as well as several other parts of the brain.
\end{itemize}
This is clearly what we consider a normally human-only behavior, but there's been studies that indicate that apes, dogs and rodents have been observed to take action at the presence of distress signals, either from humans or other members of their own species \citep{rf2}.
If this is true, theoretically, we could teach a machine learning algorithm to process signals of distress and react accordingly.

\section{Motivation}
At first, I wanted to create an algorithm that could serve as a makeshift therapy chatbot that people could use when they were confused about their own feelings, but as time has passed, a lot of things have happened in my life regarding people with close-to-none empathy.
This project could prove especially useful towards people who have trouble discerning when to console someone or having an idea of how other people or even themselves feel, such as the case of people with Asperger's Syndrome or other forms of high-functioning autism.
To this end, I've decided to work on this project.

\section{Hypothesis}
Empathy consists in a pattern of neurochemical reactions triggered by different situations. Machine learning could learn to identify these patterns without them being processed biologically.

\section{Objectives}
In this section, I will establish the objectives I would like to touch upon in this paper.

\subsection{General Objectives}
The objective of this project is to determine how the user's feeling according to the words in the input. This will be achieved thanks to the technology present in machine learning algorithms and an extensive amount of datasets.

\subsection{Specific Objectives}
\begin{itemize}
	\item Generating an algorithm capable of detecting key words related to the user's mood.
	\item Predicting successfully the user's mood according to their input.
\end{itemize}

\section{Structure}
---Work in progress---
\clearpage